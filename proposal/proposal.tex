\documentclass[a4paper]{article}

\usepackage{natbib}
\usepackage{paralist}
\usepackage{hyperref}
\usepackage{color}
\usepackage[margin, draft]{fixme}
\fxsetup{theme=color,mode=multiuser}
\FXRegisterAuthor{gp}{agp}{GP}

\newcommand{\gpn}[2][]{\gpnote*{#1}{#2}}
\newcommand{\gp}[1]{\gpnote{#1}}

\newcommand{\armour}{{\sc Armour}}
\newcommand{\ZT}{{\sc ZT}}
\newcommand{\rot}{RoT}
\newcommand{\psa}{PSA}

\bibliographystyle{plain}

\title{\armour{}: Smart Secure Policies}
\date{}

\begin{document}

\maketitle

\section{Brief}

This project will develop an infrastructure to provide data protection
for loosely coupled networked applications.
%
The infrastructure will be supported by a security policy language and
a runtime control-plane to realize data-driven application level
access control.

\section{Motivation}

Arm is heavily invested in the internet services and IoT lines of
business.
%
This is founded on a strong reputation for managing hardware/software
security.
%
As illustrated by the Healthcare PoC, data management and data
security are fundamental requirements to enable the development of
secure IoT applications.
%
Developing advanced, Arm-centred solutions for managing the security
of software services will provide Arm with a competitive advantage, or
at the very least avoid being at a competitive disadvantage with
respect to emerging x86-based cloud solutions.


\section{Problem Statement}

Modern software development makes use of technologies (such as
web-frameworks) that were developed for cloud computing.
%
Traditional monolithic applications are being replaced with
micro-service style applications, which communicate using protocols
such has REST over HTTP, gRPC, MQTT and ZeroMQ.
%
Services are often containerized and managed through sophisticated
orchestration infrastructure, using tools such as Docker and
Kubernetes.
%
Crucially, these services might be the gatekeepers for high value data
assets, such as financial and medical records.
%
The use of distributed service style architectures leads to complex
and challenging security issues, especially in multi-tenant and
multi-actor scenarios, where there will be interactions with
communicating services that could be compromised and/or malicious.
%
Furthermore, services can be internet facing, which means that they
are potentially exposed to highly-sophisticated attacks from
well-resourced parties.

\subsection{Cloud-based Solutions Assumptions}
\label{sec:cloud-sol}

A number of network security solutions exist within the cloud
micro-services space.
%
Importantly, most cloud-based applications make certain trust
assumptions:
\begin{itemize}
\item {\sc Trusted infrastructure:}
  In most cloud-based micro-services applications the execution
  platform is trusted.
  This is often achieved by having in-premises clouds or by relying on
  well-established cloud providers like Amazon AWS or Microsoft Azure.
\item {\sc Cluster Orchestration:}
  Typically the management of deployment and maintenance of cloud
  micro-services relies on orchestration software.
  %
  Examples are Kubernetes~\cite{kubernetes} for containerized
  micro-services, or Amazon Elastic Container Service
  (ECS)~\cite{aws-ecs}, Azure Kuberentes Service
  (AKS)~\cite{azure-aks}, Amazon
  CloudFormation~\cite{aws-cloudformation} for infrastructure, etc.
  %
  Among other issues these frameworks have to provide facilities for
  network communication and some of them support plugins for network
  security.
  %
  Considering only the Kubernetes orchestration as a notable example
  we find a multitude of network Container Network Interface (CNI)
  plugins, including among others Cilium~\cite{cilium},
  Calico~\cite{calico}, Flannel~\cite{flannel}, etc. 
\item {\sc Network Security:}
  As mentioned above, most of these network plugins provide some
  degree of security enforcement.
  These range from simple network connectivity (Layers 2 and 3 of the
  OSI stack), to application level APIs (Layer 7) such as the case of
  Cilium~\cite{cilium}. 
\end{itemize}

\subsection{Representative Cloud Network Security Solutions}
\label{sec:repr-cloud-netw}

Notable examples of cloud Network Security Solutions are  are Cilium,
Envoy and Istio.

\paragraph{\sc Cilium:} Cilium~\cite{cilium} focuses on high
performance API/network-based security; however, it has a few
shortcomings generally shared by all existing solutions:
\begin{inparaenum}
\item The policies are relatively simple, with rules being limited to
  the selection of end-points (service entry-points) and the API
  method called called, as well as role-based access control. 
\item It does not natively support popular publish-subscribe protocols
  such as MQTT.
  %
  (For instance extensively used in the Healthcare PoC.)
\item The codebase for Cilium is large and complex, and there are many
  software dependencies.
  %
  Currently it has not been made to work on Arm hardware.
\item Being a CNI plugin for Kuberenetes, it shares its
  control-plane/data-plane model, whereby a master controls an agent
  in each node, and it centralizes the configuration of security among
  other services.
\end{inparaenum}

\paragraph{\sc Istio:} This~\cite{istio} is another interesting
framework.
% 
Istio provides a mesh-like network infrastructure with emphasis on
security.
%
In particular Istio enforces the security of the underlying
communication channels, and it provides implementations of secure
naming, communications backed by strong cryptography, key and
certificate management, etc.
%
Unlike Cilium, Istio network security policies are limited to L2/L3. 
%
Istio is a mature and widely deployed tool, but much like Cilium,
Istio is complex and it is not currently Arm compatible.
%
Like the Kubernetes master in the case of Cilium, Istio's
control-plane centralizes the configuration and deployment of
policies, which are used to set up the proxies in the nodes of the
mesh implementing the policy.

\paragraph{\sc Envoy:} Envoy~\cite{envoy} is an open source edge and
service (reverse)-proxy focused on information security.
%
Importantly, both Cilium and Istio use Envoy to implement their
security policies.
%
Envoy can route HTTP methods, configure routing and enforce L2/L3
policies.
%
However, Envoy is only a proxy, so there is no control-plane to manage
a complete distributed system. 

\paragraph{\sc Other Solutions:}
There is a multitude of solutions in this space, but most of them have
similar features and shortcomings.
%
In particular, it is not clear that these solutions are well adapted
to the case of edge/IoT applications, where the assumptions on trust
on the infrastructure are not necessarily well established. 

\paragraph{\sc Going Beyond API filters:}
In this project we will cherry-pick some key features of Cilium, Istio
and other existing solutions, and at the same time support the use of
\emph{more expressive security policies}, allowing rules of the form
\emph{“based on what we have observed before, service A is currently
  allowed to make request R to service B”}.

Beyond the network connectivity and the assumptions of trust on the
infrastructure, two important \emph{features lacking} in all of the
existing frameworks are the capabilities to
\begin{enumerate}
\item track \emph{sessions} and application-level protocols, and
\item to \emph{inspect the data} being protected (information flow audit,
  leak prevention) to the extent prescribed by the security policy.
\end{enumerate}
In the absence of these features, the enforcement of data and
session security has to be relegated to the application(s) developer(s).
%
However this is problematic in the case of applications developed by
different parties as illustrated by the security requirements of the
health-care PoC.
%
This project will provide the policy writer with the capability to
address these issues at the policy level instead of relying on well
behaved applications.

\paragraph{\sc Incremental Security:}
An important aspect of our proposed solution is that it should not
interfere with the security measures put in place in the original
application developer.
%
Consider for instance the case where the application developer has
implemented secure communication channels among her/his trusted
micro-services.
%
It is completely acceptable to have an \armour{} policy that allows
these channels to exist (for which neither data nor session tracking
can be implemented by \armour{}).
%
However, the \armour{} policy should be permissive enough to allow
unrestricted communication between these end-points, which would be
presumably the case if the application developer takes part in the
policy specification. 
%
On the other hand, if no such security has been implemented by the
application developer, but it is required by the security policy,
\armour{}s implementation has the capability of enforcing it to
achieve the desired policy.

To reiterate, \armour{} policies do not require that all communication
between any two services be made visible to the implementation
engine, but if any ``encrypted'' channels exist, they should not
contradict the policy.

% \gp{Add note about encryption at this point.}

\subsection{Tailoring to Different Trust Modes}
\label{sec:attack-surface}

\begin{table}[t]
  \centering
  \small
    \begin{tabular}[t]{| c | r | c | c | c | c | c |}
      \hline
         & Use-Case            & Devs.   & Apps.  & Policy        & App.      & Infra.    \\
      \hline
      1. & 1 Tenant            & 1 Org.  & 1 App. & Centralized   & Trusted   & Trusted   \\
      2. & N Trusted tenants   & N Orgs. & M Apps & Centralized   & Trusted   & Trusted   \\
      3. & N Untrusted tenants & N Orgs. & M Apps & Centralized   & Untrusted & Trusted   \\
      4. & N Untrusted tenants & N Orgs. & M Apps & Decentralized & Untrusted & Untrusted \\
      \hline
    \end{tabular}
    \caption{Trust modes}
    \label{tab:attacker}
  \end{table}


% \begin{figure}[t]
%   \centering

    
%   \caption{Different Security Assumptions}
%   \label{fig:attacker}
% \end{figure}


An important aspect of any security solution for multi-party
distributed systems is the consideration of the trust model, as well
as the assumptions that can be made of the infrastructure running the
application.
%
Here we refer by ``trust'' to the assumptions that components of an
application can make about the intent of other components, that is,
whether they are benign or malicious.
%
We do never assume that code is free of bugs or security holes. 
%

Because the capability of \armour{} to enforce security policies will
be restricted by the trust model of the underlying application and
infrastructure, we aim to gradually and incrementally consider the
use-cases presented in~\autoref{tab:attacker}:
\begin{enumerate}
  % \gp{Verification Monitor case}
\item In the fist case we consider an application written by a group
  of developers belonging to a single organization.
  %
  We consider a distributed application for which communication
  security might not have been put in place by the application
  developer.
  %
  In this case, a centralized network security policy would establish
  the security requirements.
  % 
  This is typically the case of applications running in cloud
  environments, and are typically the clients of mesh-like security
  solutions such as Istio~\cite{istio} and Envoy~\cite{envoy}.
  %
  % In the simplest case, if the policy relates only to network security
  % or REST APIs tools like Cilium~\cite{cilium} could also be applied.

  % \gp{Clarify that this does not mean security/bug free.}
  Importantly, the security assumptions of this kind of application
  are that the application code is itself trusted by all the
  components (since there is a single organization), and that the
  infrastructure is also trusted (in premise, or a trusted provide
  such as AWS).

  For an application of this type, the use of \armour{} would provide
  security to the communication channels, and it would validate that
  the application communication respects the security policy, for
  instance to prevent information leaks due to application errors,
  design inconsistencies, unsafe libraries potentially opening
  unauthorized communication channels, etc.
 
\item In the second case we find applications that incorporate
  components of different organizations or providers.
  %
  This could for instance be a simplified case of the health-care
  PoC where all the containerized applications are assumed to trust
  each other.
  %
  Importantly, in this case a centralized policy could validate or
  enforce that security measures are respected.
  %
  This is particularly important because when combining software
  components from different providers it is not immediately obvious
  what the composed behavior of the application is, nor whether it
  respects high-level policies.
  %
  For instance, can we easily know when the addition of a new
  pharmaceutical company to the healthcare PoC will respect the
  patient's security requirements? This is not trivial even in the case
  where the pharmaceutical application is trusted since it requires
  the consideration of the behavior of the whole application.

  In this scenario the \armour{} policy would act as a safety-net
  ensuring that the composition of different (perhaps dynamic) actors
  does not violate the high-level security objective.
  
  The trust model for this scenario requires that all applications
  trust the enforcer of the security policy, and the policy acts as a
  security monitor to prevent errors stemming from the composition
  with unknown code.
  %
  % The applications do not-necessarily protect from dishonest
  % co-tenants.

\item The third use-case we consider best corresponds with the
  assumptions made by the healthcare PoC application, where we assume
  that Arm manages the compute infrastructure, and a number of
  different organizations run different components of the application.
  %
  We take this case as a typical baseline for IoT-like systems where new
  devices or services are added on-demand from a set of
  not necessarily trusted providers.

  Thus, we can consider that while the application providers do not
  trust each other, they do trust the infrastructure.
  %
  In that sense, having a centralized policy run by the infrastructure
  provider could achieve the desired security objectives for all the
  parties.
  %
  This is typically the case of a cloud-provider, where the tenants do
  not necessarily trust each other, but they do trust the provider
  (isolation, networking, VPNs, VPCs, etc.).
  %
  Hence, the security assumptions in this case are: the different
  applications do not trust each other, the applications do trust the
  infrastructure, and therefore if the \armour{} security policy is
  managed by the infrastructure provider, the applications can rely on
  this security policy to achieve trust. 
  
\item The final and most adversarial case is one where there is no trust
  among the applications, nor is there trust in the infrastructure
  provider.
  %
  One can argue that the application developers do not need
  to trust the infrastructure provider with their data, and hence the
  application of proxies that inspect data and keep track of sessions
  is not applicable to this case.
  %
  This is however not incompatible with the objectives of \armour{}. 
  
  For this kind of adversarial environment we will consider the
  implementation of a network/communication substrate supported by
  Trusted Execution Environments (TEEs) to provide end-to-end privacy
  preservation guarantees.
  %
  For instance, we could consider TEE-backed proxies or filters such
  that the different participating parties can remotely attest their
  correctness (cf. remote attestation).
  %
  Upon successful attestation the parties can trust that the only
  function of these proxies is to enforce the agreed upon security
  policy, without ever deviating from their ``verified'' expected
  behavior, and hence not revealing any ``secrets''. 
  %
  This is the most speculative thrust of the project, and it remains
  largely an open research problem that we would like to explore,
  where we believe we can make substantial contributions. 

  Since in this case a central policy coordinator might not be
  achievable, we will consider that the security policies might have to
  be decentralized, potentially with each party providing its own, and
  that Trusted Execution Environments (TEEs) will have to be leveraged
  to ensure security of the individual services.
  %
  Possible implementations of these TEEs include but are not limited
  to OS-isolation, SGX Enclaves or Newmore realms, etc.
\end{enumerate}


\section{Identity Management and Authentication}
\label{sec:identity-management}

This section is motivated (in part) by the Zero Trust Networks (\ZT{})
approach to network design and protection~\cite{GilmanB17}, and it
attempts to leverage notions of identity supported by Root of Trust
(\rot{}) mechanisms advocated by Arm and the Platform Security
Architecture (\psa{})~\cite{PSA}.

\ZT{} networks recommends that all notions of identity should be
rooted in PKI certificates.
%
Different entities to which certificates can be associated are:
\begin{inparaenum}
\item Devices, 
\item Users, and
\item Applications. 
\end{inparaenum}

Device and application trust should be supported by an attestation
procedure allowing to associate concrete hardware devices with logical
identifiers in \armour{} for the former case, and associate software
measurements provided through attestation to logical application
instance identifiers in \armour{} for the latter case.
%
The notion of a user is dependent on the infrastructure (Argus, AWS,
GC, etc.), and it can also be application-specific.
%
Because of that, it is perhaps best to handle user authentication by
means of \armour{} oracles, although this needs to be studied.

While these notions of identity are directly related to a
physical/measurable entity, there will be other notions of identity
which will be fundamentally logical.
%
Examples of these are
\begin{inparaitem}[-]
\item Kubernetes Services, Roles, Pod names, 
\item AWS security groups, AIM,
\item Unix user and group names, etc.
\end{inparaitem}
%

Another important aspect of authentication is that some logical
entities are persistent (devices, users, etc.), while other entities
are volatile (workloads, FaaS instances, temporary services, etc.).
%
\armour{} will requires that all logical notions of identity be
backed by entities that have been authenticated using PKI
certificates. 
%
To that end, \armour{} will need to provide mechanisms to associate
the identity of logical and/or volatile entities to the identity of
physical or persistent entities that can be attested through PKI
certificates.
%
In some sense, this association propagates the \rot{} from the
attestation mechanisms provided by hardware (eg. \psa{}) to the
high-level identity primitives manipulated by \armour{}.

\subsection{From \rot{} to \armour{} IDs}
\label{sec:from-rot-armour}


\section{Identities and Authorization}

Again, following the \ZT{}~\cite{GilmanB17} framework, it would be
ideal to make the authorization and authentication management of an
\armour{} policy independent.
%
This however does not mean that they can influence each other.
%
In particular, it is expected that most authorization policies will
need to refer to aspects of authentication, even if indirectly through
the logical identities established.

\gpn[TODO]{
  This that should be discussed somewhere. Needs design decisions. 
  \begin{itemize}
  \item Through and through encryption, probably supported by mTLS/IPSec,
  \item Authentication (TODO) vs. authorization (current policy language), 
  \item Do we need network agents a la \ZT{}?, 
  \item From \rot{} device + user identity to workflow? service? channel?, 
  \item Proxy management, data/control plane API + security,
  \item Who signs what? Is control plane self-signing and it signs
    everything else?, 
  \item Implementation of certificate authority a la Citadel?, 
  \item Policy distribution? A la Kubernetes? Replication?
    Consistency?, 
  \item TLS vs IPSec?, we might be able to chose.
  \end{itemize}
}

\section{Design Principles}
\label{sec:design-principles}

\begin{table}[t]
  \centering
  \begin{tabular}[t]{l|l}
    \multicolumn{1}{c}{Security}
    & \multicolumn{1}{c}{Extensibility}\\
    % & \multicolumn{1}{c}{Expressivity}\\
    \hline
    Memory Safety (Rust)          & Oracle policy extension \\
    Defensive programming (Sanitization) & Oracle types            \\
    Lightweight Types             & Service Types           \\
    Service compartmentalization  & User-defined Roles      \\
    Control/Data plane separation &                         \\
    Native RoT verification       & \\
    mTLS throughout & \\
    Cryptography-backed services (enclaves?) &
  \end{tabular}
  \caption{Design Considerations}
  \label{tab:design}
\end{table}

The main design principles driving \armour{} are detailed
in~\autoref{tab:design}.

The main principles driving the security of \armour{}s implementation
are:
\begin{inparaenum}
\item \textbf{defensive programming},
\item \textbf{process/service isolation}, and 
\item \textbf{Cryptography throughout}. 
\end{inparaenum}

\gpn{It is important to separate which are the programming practices
  that make the \armour{} implementation secure, vs. which are the
  security features of \armour{}.}

VSFTP design principles~\cite{VSFTP}.

%
% \begin{itemize}
% \item Security by construction design
%   \begin{itemize}
%   \item Memory-safety from Rust
%   \item Sanitization of inputs
%   \item Lightweight typing to avoid common errors at runtime
%   \item Compartmentalization of functionality in different services to
%     disallow privilege escalation
%   \item Separation of control and data-plane functionalities
%   \end{itemize}
% \item Extensibility and flexibility by design
%   \begin{itemize}
%   \item The oracle model allows for arbitrary policy extensions
%   \item Policy compartmentalization can be achieved by either
%     \begin{inparaenum}
%     \item providing different policies to different entities, or
%     \item by implementing component-specific oracles
%     \end{inparaenum}
%   \end{itemize}
% \end{itemize}

\section{Vision}

When this project is completed, Arm will have access to prototype
proxying middleware that runs on Arm hardware and provides advanced
policy-based security features for distributed micro-service style
applications.


% \section{Research Focus}

% Security, Internet Services, Networking, Operating Systems 

% \section{Tech Transfer}

% \section{Status Summary}

% \section{Project Summary}

\bibliography{proposal}

\end{document}

%%% Local Variables:
%%% mode: latex
%%% TeX-master: t
%%% End:
